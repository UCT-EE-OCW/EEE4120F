\section*{Introduction}
Welcome to the practicals for EEE4120F. Please take note of these instruction that are applicable to all practicals in this course.

Practicals, with tutors to assist, generally take place on a Thursday, from 14h00 to 16h00; but please check the schedule and announcements on Vula as the schedule may change. You are responsible for completing these practicals in order to pass this course. Some of these practicals may require access to specialist hardware that will only be available in labs during the practical session - however, for 2021 we have tried to plan all pracs needed for this course to require nothing other than a decent computer, ideally a computer with a nVidia GPU which has CUDA and OpenCL support (most recent nVida GPUs provide such support). The FPGA pracs have been reworked so that they can be done using simulation, thus you do not need to make use of labs for even those practicals which should enable you to pass the course without needing to use labs on campus if you would prefer not to.

It is assumed you have knowledge regarding tools such as git, \LaTeX{} and running programs from the Linux command line. If you are wanting to do the pracs on your own computer, and don't have Linux installed, you are suggested to install Ubuntu Linux either as a dual boot option or to run it using virtualization, e.g. \href{https://www.virtualbox.org/}{VirtualBox}.

It is suggested you have a dual boot system. The reason for this is that development tools are generally better on a Linux based system, whereas proprietary tools are often better supported on Windows. We recommend Ubuntu 18.04 LTS, though swapping between Ubuntu (or any Linux distro) and Windows is an effective strategy, particularly when using proprietary software.

The source code for all pracs is available on the EE-OCW GitHub Project Page: \href{https://github.com/UCT-EE-OCW/EEE4120F-Pracs}{https://github.com/UCT-EE-OCW/EEE4120F-Pracs}. You can download the source code for pracs from there.