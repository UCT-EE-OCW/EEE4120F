\section{Milestones}
\label{sec:milestones}
The project consists of the following sequence of milestones (MS), from MS1 to MS6. The details of each milestone are covered in this Section. For a quick overview, refer to Section \ref{sec:overview_milestones} or Table \ref{tbl:YODAMarkTotal}. Some of these milestones will run in parallel - it is your responsibility to be aware of the due dates.

\subsection{Group and Project Selection}
\label{ms:teamupdate}
The starting point of the YODA project is to either: 
\begin{enumerate}
    \item to go about finding people (another two people to be precise) to work with and then decide as a group which project to go for, or
    \item to choose a topic you want to do and try to find a group to join and risk not finding a group.
\end{enumerate}

It perhaps depends on your personality but a bit of social interaction might be an effective starting point!


\subsubsection{Group Selection} 
Ideally, groups of three should be formed. Groups of 2-4 may be selected amongst yourselves, but you need to notify the lecturer to see potential addition/reduction of your project workload according to your team size.

Each team will be assigned a team manager. This is someone who is not in the team, but rather a tutor, TA or lecturer who will act both as the manager and something of a consultant. Please be conscious of your ``team manager's" time constraints - you can't bug them with many questions; you are meant to be largely responsible for your own development practices and problem-solving.

\subsubsection{Topic Selection}
The procedure for selecting a topic is to start by looking over the available topics that are offered in this booklet. If you are inspired to suggest your own idea for a topic, you are most welcome to do so (see Section \ref{sec:SelectOwnTopic} if so). Topic allocation is first come first serve. Take note of the topic number of the project you want to work on. If you don't see any topic that interests you, consider looking online or at some books in the library.

You may want to do some web searching or textbook reading to better understand the topics and understand the extent of work that may be needed.

Some of the topics are more work than others, but most are planned around being doable by a team of 3 given the time constraints for this course project.

It is indicated in the project descriptions if it is a harder than average project - if you and your teammates are particularly ambitious and want an opportunity to really develop your HDL skills, then it's certainly recommendable to try going for one of the more demanding projects.


\subsubsection{Selecting a Standard Topic}
The topic selection process is self administrated. While we would like to do first come first serve, if you would like to select the same topic as another group, you can propose some changes. If there is enough of a difference you may be allowed to undertake the project. You will need to email the lecturer to get the choice authorised. First check the Vula site to see that no other group has selected the topic your group is planning on. If the topic is already selected, then sorry; please try another option. Alternatively, select your own topic as described in Section \ref{sec:SelectOwnTopic}.

\subsubsection{Selecting Your Own Idea as a Topic}
\label{sec:SelectOwnTopic}
In the case that you want to do your own topic idea, first prepare a short abstract describing the topic. It should follow a structure as given in the topic description on this page. You need 1) a topic title, 2) a textual description of the topic, 3) explanation of inputs and outputs. You can put in figure(s) to help explain the topic, and if you like specific optional additions that might be added if time permits (of course describing the optional additions doesn't mean you will commit to doing them, it is just nice to think about how the topic could be further developed should there be time or interest). It is expected that the project topics can be put in the public domain (if for any reason you disagree with this the please discuss this with the lecture via email or in person to discuss the issue and negotiate a compromise).

Send your abstract to the lecturer for approval and possible modification (i.e., to ensure it has sufficient work at a suitably technical level and won't take too long to complete). Once you have approval you'll be assigned a topic number; then please go ahead and follow the step to record your project team in the Yoda Topic Groups page on the Vula site for this course.

\subsubsection{Submission}
To complete this milestone, your team leader will need to fill out a Google Form, which will be linked to you closer to the due date. On this form you will need to submit your team members and topic selection. If you are proposing your own topic, you will need to fill out those details, too.

\subsection{Conceptual Design}
\label{ms:conceptualdesign}
This is an in person meeting with a tutor, TA or lecturer. You will present snippets of code, design documents (UML suggested), and some initial prototyping or simulation. It is an early stages meeting, unlike the Status Update. This is for marks. The rubric can be seen in Table \ref{tbl:DesignReview}

You will also need to upload a supporting document to Vula containing:
\begin{itemize}
    \item role allocation (of the role names are up to you to decide you can apply your mind to come up with something more exciting than team leader)
    \item division of labour plan
    \item concept sketch (at least one)
\end{itemize}
.

\begin{table}[H]
\centering
\caption{Mark Sheet for MS2: Conceptual Design}
\label{tbl:DesignReview}
\begin{tabular}{|l|l|}
\hline
\textbf{Task Description} & \textbf{Max} \\ \hline
\begin{tabular}[c]{@{}l@{}}Identify information sources used in coming up with the \\ proposed design.\end{tabular} & 5  \\ \hline
\begin{tabular}[c]{@{}l@{}}Identify important skills needed to complete this YODA \\ project.\end{tabular} & 5  \\ \hline
\begin{tabular}[c]{@{}l@{}}Describe the methodology to develop the proposed \\ YODA product. Describe two approaches: plan B can be \\ a modified/simplified version of the first approach plan A.\end{tabular} & 10  \\ \hline
\begin{tabular}[c]{@{}l@{}}Modelling and analysis. Present a (draft) model of your YODA\\  product design. This can be in the form of a paperbased (or\\  digital) block diagram or UML model.\end{tabular} & 10  \\ \hline
\begin{tabular}[c]{@{}l@{}}Discuss the strategy for evaluating your design solutions \\ using paper-based calculations and estimations of \\ performance criteria.\end{tabular} & 10  \\ \hline
Use of visual aids / sketches / whiteboard & 10  \\ \hline
\textbf{TOTAL} & \textbf{50}  \\ \hline
\end{tabular}
\end{table}

\subsection{Status Update}
\label{ms:statusupdate}
You need to update your assigned team manager, i.e. the tutor or lecture to whom you have been assigned to report on your progress. This needs to be done in a short in-person meeting, possibly with a follow-up where you post the substantiating evidence of your status report on this Vula assignment. 

NB: upload supporting document, must provide:
\begin{itemize}
    \item Design document  (block diagrams, schematics)
    \item Snippets of code
    \item Some initial simulation (if not real prototyping) results
\end{itemize}

You can add other things you think would demonstrate good progress.

The purpose of this milestone is to show progress that you are making on the YODA project. Please meet with your team manager or email your manager to provide an update of your progress. Mention at lease 3x things that you have already done on the project (in terms of design work), 3x things that you are planning to do over the next couple of days (also design/implementation) and show some evidence of work done (if in person show on a PC or by email you can send screenshots of attached code files or figures). We will also award a quality of the work, for instance if your code is well structure and commented or your schematics well labelled you will get a higher value for the quality mark.

This is for marks. The rubric can be seen in Table \ref{tbl:StatusUpdate}.
\begin{table}[H]
\centering
\caption{Mark Sheet for MS3: Status Update}
\label{tbl:StatusUpdate}
\begin{tabular}{|l|l|}
\hline
\textbf{Task Description} & \textbf{Max} \\ \hline
Overview of Progress & 10  \\ \hline
3x Things Done & 5   \\ \hline
3x Things to be done & 5   \\ \hline
Design document or explanation of your code/design so far & 10   \\ \hline
Evidence of work done (e.g. code snippet) & 10   \\ \hline
Overall quality & 10   \\ \hline
\textbf{Total} & \textbf{50}   \\ \hline
\end{tabular}
\end{table}

\subsection{Draft YODA Paper}
\label{ms:draftpaper}
As mentioned in the introduction to the YODA project, this project experience is planned around mimicking a professional/academic experience whereby you develop a product or conduct an investigation (i.e. do the YODA project) and then you plan on presenting it to the `community' at a conference to both show them what wonderful things you've done (get your moment in the limelight) but also get their take an impressions - as well as marking the idea to get the chance of other researchers/developers using your work (and for the academic case getting references or for the industry case getting e.g. people to buy licenses to use your system/IP).  

The `community' we are dealing with here is other YODA system developers. The usual process with a good quality conference (e.g. an IEEE or ACM conference) is:
\begin{enumerate}
    \item A first round where authors submit short papers (usually not more than 6 pages for IEEE) to the conference.
    \item These are then reviewed by members of the reviewer committee (these are experts in the field why generally give of their time to review and rate papers submitted to the conference).
    \item If one is lucky and the reviewer(s) of your paper approve it for the conference, they are likely to request corrections to improve the descriptions and quality. Most conferences need you to fix and resubmit the paper before you are accepted to present it... but other may give authors a bit of time after the paper is presented to finish revising and to submit the paper  (this approach is sometimes only done if there is a journal-link to the conference where authors of particularly good papers are invited to submit to the journal).
\end{enumerate}

In our case the authors will be given a few days after the paper is presented to submit it to the `conference proceedings' (actually just submitting on Vula to be marked as a concise project report).

Those students getting top rating for the YODA project will be invited to have an entry in the \href{http://ocw.ee.uct.ac.za/courses/EEE4120F/HOF.html}{Hall of Fame} on the EEE4120F website.


For this milestone, you must submit a draft paper (in the IEEE conference format) to Vula. This is to show that you are making progress. You should have something to write for each part of the report through into to conclusion, but it obviously doesn't need to be fully complete, e.g. can indicate where you plan to put additional result images once you're done the needed tests.

While this does not count for marks, it may likely improve your final report submission if you provide the draft report.

You can find hints on using \LaTeX \space on the \href{http://wiki.ee.uct.ac.za/LaTeX}{UCT EE Wiki}.

\subsection{Conference Presentation}
\label{ms:conferencepresentation}
The YODA mini conference is scheduled for the last week of lectures in term 2, which provides an opportunity for each group to present to the other students, and any invited guest, on their YODA project. This is meant as a simulated professional conference where each team will have a short period of time to present their projects - it is required to have slides to present and design details and optional to have a live demo of the prototyped system. The demos should be shown to the lecturer and/or alternate marker (usually after the YODA conference). The conference is scheduled for the afternoon lecture on Tuesday and the double period on Thursday, the timing has to be quite precise. Each student needs to attend at least one other presentation session besides their own so as to provide an audience for another group's presentation. Attendance marks will be allocated, a large part of the project marks is allocated to the YODA presentation.

You will need to book a presentation slot on Vula closer to the time.

The mark sheet is shown in Table \ref{tbl:ConferencePresentation} below. Try and follow a logical approach, and ensure your presentation contains the five essential elements of a typical design and evaluation reporting presentation:
\begin{enumerate}
    \item Introduction \& Problem Description
    \item Methodology / Approach taken
    \item Design
    \item Experiment and Results
    \item Conclusions \& Questions
\end{enumerate}

% Please add the following required packages to your document preamble:
% \usepackage{multirow}
\begin{table}[H]
\centering
\caption{Mark Sheet for MS5: Conference Presentation}
\label{tbl:ConferencePresentation}
\begin{tabular}{|l|l|l|}
\hline
\textbf{Section} & \textbf{Description} & \textbf{Max} \\ \hline
\begin{tabular}[c]{@{}l@{}}A Introduction \& Problem \\ Description\end{tabular} & Recap of concept & 3 \\ \hline
 & Problem Description & 3 \\ \hline
 & Overview of prototype design (main components) & 3 \\ \hline
 & Visual aids (images \& pointing to things, etc.) & 6 \\ \hline
B  Methodology/Approach taken & How was the problem approached, steps taken & \multirow{2}{*}{10} \\ \cline{1-2}
 & \begin{tabular}[c]{@{}l@{}}How will it be tested (e.g. digital accelerator)\\ Explained tools / equipment / hardware used\end{tabular} &  \\ \hline
 & Expected Results & 5 \\ \hline
C Design & Presentation of the design & 12 \\ \hline
 & Visuals, appropriate amt. detail, clarity & 10 \\ \hline
D Experimentation and Results & Explanation \& showing test inputs & \multirow{2}{*}{15} \\ \cline{1-2}
 & Display of result &  \\ \hline
 & Comparison to baseline / gold measure & 7 \\ \hline
E Conclusions andQuestions & Considered experiment success/failure & 3 \\ \hline
 & Ways to improve results, future work suggestions & 3 \\ \hline
 & Handling of Questions & 8 \\ \hline
F Participation and Preperation & \begin{tabular}[c]{@{}l@{}}Adequately prepared, each member allocated \\ something to do and knows what to do.\end{tabular} & 12 \\ \hline
 & TOTAL & 100 \\ \hline
\end{tabular}
\end{table}

\subsection{Final Hand In}
\label{ms:finalhandin}
This is the submission for the final YODA report. The YODA final submission counts for 45\% of the project. The report count 40\% and the code/design repository counts 5\%. Marking scheme can be seen in Table \ref{tbl:FinalReport}. Each category will be marked along the guidelines of the marking rubric presented in Table \ref{tbl:FinalRubric}. For example, if the abstract is fantastic it would get 10/10, but if it is only mediocre or worse it would unlikely get more than 5/10.

\begin{table}[H]
\centering
\caption{Mark Sheet for MS6: Final Report}
\label{tbl:FinalReport}
\begin{tabular}{|l|l|l|}
\hline
\textbf{Section} & \textbf{Description} & \textbf{Max} \\ \hline
Abstract & About 250 words, giving an overview of your project. & 10 \\ \hline
Introduction & \begin{tabular}[c]{@{}l@{}}A nice lead-in that states objectives and motivations\\ clearly\end{tabular} & 10 \\ \hline
Background & \begin{tabular}[c]{@{}l@{}}Points supporting/leading to the motivation, some\\ mention of literature or sources of information\\ inspiration (textbooks and online repositories)\end{tabular} & 5 \\ \hline
Methodology & \begin{tabular}[c]{@{}l@{}}How you proposed to go about building the system,\\ some of which should have managed to do, but\\ much of this may be hypothetical/recommended\\ (i.e. if there was time, describe what you would do).\\ Discuss how you would measure the results, and\\ what measurements/metrics you would record, so\\ there is a scientific grounding to your study.\end{tabular} & 10 \\ \hline
Design & \begin{tabular}[c]{@{}l@{}}Elaborate on the design of the accelerator system\\ and give thoughts on connecting up to a host. (note:\\ methodology and design are not the same thing)\end{tabular} & 20 \\ \hline
\begin{tabular}[c]{@{}l@{}}Proposed \\ Development \\ Strategy\end{tabular} & \begin{tabular}[c]{@{}l@{}}This can be conceptual/something of a thought\\ experiment. Discuss a bit as to, if this was a\\ commercial product, what sort of supporting tools\\ and framework would be needed to facilitate\\ application development using this accelerator.\end{tabular} & 10 \\ \hline
\begin{tabular}[c]{@{}l@{}}Planned \\ Experimentation\end{tabular} & \begin{tabular}[c]{@{}l@{}}Elaborating on the methodology, describe the\\ experimental setup, how the experiments were\\ implemented, e.g. commands performed. Could\\ explain how golden measure would be used to\\ compare accuracy of prototyped system (note results\\ section shows the actual results, this just explains\\ the experiments in more detail)\end{tabular} & 5 \\ \hline
Results & \begin{tabular}[c]{@{}l@{}}Show your results (even if it's only golden measure\\ testing).If you don't manage to get much results then\\ add discussion about what would be anticipated\\ were there time to do it (to run the proposed\\ experiments discussed in the previous section), and\\ if there is no time to complete experiments then\\ provide model graphs providing an example of what\\ type of performance / output / other results would be\\ expected and an argument as to why you would expect\\ these results.\end{tabular} & 15 \\ \hline
Conclusion & \begin{tabular}[c]{@{}l@{}}Summarize the results collected. Were the objectives\\ discussed in your design achieved? What else can\\ be done to improve the system going forward?\end{tabular} & 10 \\ \hline
References &  & 5 \\ \hline
 & Total & 100 \\ \hline
\end{tabular}
\end{table}

\begin{table}[H]
\centering
\caption{Rubric Guide for MS6: Final Report}
\label{tbl:FinalRubric}
\begin{tabular}{|l|l|}
\hline
\textbf{Marking Rubric} & \textbf{Percentage Range} \\ \hline
\begin{tabular}[c]{@{}l@{}}Totally fantastic and going far beyond what was\\ anticipated\end{tabular} & 90-100 \\ \hline
\begin{tabular}[c]{@{}l@{}}Significant effort and very well done, maybe slight\\ improvement\end{tabular} & 75-89 \\ \hline
\begin{tabular}[c]{@{}l@{}}A good effort, not quite what was anticipated, minor\\ aspects lacking or not clear enough\end{tabular} & 70-74 \\ \hline
Some effort but a bit inconsistent, or lacking in parts & 60-69 \\ \hline
\begin{tabular}[c]{@{}l@{}}Mediocre, parts incomplete or handled only\\ superficially\end{tabular} & 50-59 \\ \hline
Totally inadequate, lacking major aspects & \textless{}50 \\ \hline
\end{tabular}
\end{table}

\subsection{Due dates}
Deadlines for the tasks above will be given on Vula.