\section{Introduction}
\textbf{Y}our \textbf{O}wn \textbf{D}igital \textbf{A}ccelerator (YODA) is designed to test your design and implementation ability when creating digital accelerators. A digital accelerator is a piece of hardware used to run an algorithm that is usually implemented in software. Certain algorithms (not all of them!) are faster in hardware. Using an FPGA allows us to prototype hardware implementations of these algorithms, and, oftentimes, provide a speed up over execution of software on a general purpose CPU.

This project experience is planned around mimicking a professional/academic experience whereby you develop a product or conduct an investigation (i.e. do the YODA project) and then you present it to the 'community' at a conference.

The project is structured in such a manner to be easy to follow through a set of deadlines, but will require effort to implement something worthy of the Hall of Fame (HoF).

\begin{figure}[H]
\centering
\includegraphics[width=0.7\columnwidth]{Figures/yoda}
\caption{The Wise Words of a Jedi Master (probably)}
\end{figure}

The HoF is a collection of the best projects for the year\footnote{Not to be confused with The Hoff.}. The categories are Best Prototype, Best Report, and Best Concept. If you manage to enter into the HoF, you will be \href{http://ocw.ee.uct.ac.za/courses/EEE4120F/HOF.html}{immortalised on the course website}! 

\subsection{Teams}
Teams are planned around being three members. If you need to formulate a team of 2 or 4 then please notify the lecturer (before starting on the project) and an additional element of work (for a team of 4) may need to be added, or a simplification applied (for a team of 2) to make things more fair all round. For teams of three, the workload will be arranged accordingly (with the assistance of your 'team manager' - see Section \ref{ms:teamupdate}) so that each student in the class will have a similar workload in terms of time spent on this project.

\subsection{Milestones}
\label{sec:overview_milestones}
The project is broken down in to milestones, to better guide you through the project requirements. More information on these milestones can be found in Section \ref{sec:milestones}.

The following milestones are included in this project:
\begin{enumerate}
    \item MS 1 - Team update\\
    Details of who is in your team, and the project you've selected. More details are available in Section \ref{ms:teamupdate}.
    \item MS 2 - Conceptual design\\
    Consists of an in person meeting with a tutor, TA or supervisor, as well as a submission to Vula. You will present some conceptual design elements, such as some UML or code snippets. More details are available in Section \ref{ms:conceptualdesign}.
    \item MS 3 - Status update\\
    Consists of an in-person meeting and a submission to Vula. The purpose of this meeting is to show progress on your implementation. More details are available in Section \ref{ms:statusupdate}.
    \item MS 4 - Draft paper\\
    Not for marks, but it is recommended that by this stage you have started working on the final submission paper, as it will assist you with time management and the conference presentation. More details are available in Section \ref{ms:draftpaper}.
    \item MS 5 - Conference presentation\\
    A presentation at a mini-conference, showing the class the goals of your project and what you have managed to achieve. More details are available in Section \ref{ms:conferencepresentation}.
    \item MS 6 - Final paper and code submission\\
    The final hand in. More details are available in Section \ref{ms:finalhandin}.
\end{enumerate}

\subsection{Marking}
The YODA Project counts for 20\% of your final mark. The total mark break down for the project is as follows:

\begin{table}[H]
\centering
\caption{YODA Total Mark Breakdown}
\label{tbl:YODAMarkTotal}
\begin{tabular}{|l|l|r|}
\hline
\textbf{Milestone} & \textbf{Description} & \multicolumn{1}{l|}{\textbf{Mark Allocation}} \\ \hline
MS1 & Team update & 5\% \\ \hline
MS2 & Conceptual design & 15\% \\ \hline
MS3 & Status update & 10\% \\ \hline
MS4 & Draft paper & 0\% \\ \hline
MS5 & Conference presentation & 25\% \\ \hline
MS6a & Final paper & 40\% \\ \hline
MS6b & Code & 5\% \\ \hline
 & TOTAL & 100\% \\ \hline
\end{tabular}
\end{table}